\documentclass{article}

\usepackage{amsmath}
\usepackage{graphicx}

\title{Efficiency from Combined Distributions}

\begin{document}

\maketitle
\begin{abstract}
    Combining two distributions is useful
\end{abstract}

\section{Introduction}
Simulations are widely used in particle physics and science in general to test models and predictions.
One application in particular involves using a Monte-Carlo simulation (hereafter referred to as ``MC'')
to estimate the ``detector efficiency'': the function determining whether a given event will be detected.
This efficiency function may have a complicated form: in the case of a two-body decay (e.g. $D^0 \rightarrow K^+\pi^-$)
the efficiency function will be a constant; for more complicated decays it will have a dependence on some possibly
high-dimensional space. For example, the efficiency function for the $D^0\rightarrow K^+\pi^-\pi^+\pi^-$ decay will
vary over a 5-dimensional space.

A common technique for evaluating the efficiency is to compare MC simulation to
what was generated, as seen in Eq.~\ref{efficiency_definition}
\begin{equation}
    \epsilon(x) = \frac{\rm{Reco}(x)}{\rm{Gen}(x)}
    \label{efficiency_definition}
\end{equation}
Where we find our efficiency $\epsilon(x)$ by comparing our reconstructed (MC) and generated distributions.

In some cases we may have two reconstructed and generated distributions, but expect them to obey the same
efficiency function:
\begin{equation}
    \epsilon(x) = \frac{\rm{Reco^\alpha}(x)}{\rm{Gen^\alpha}(x)} = \frac{\rm{Reco^\beta}(x)}{\rm{Gen^\beta}(x)}
    \label{two distributions}
\end{equation}
where our labels $\alpha$ and $\beta$ may refer to e.g. two CP-conjugate decays.

When evaluating the efficiency using Eq.~\ref{two distributions} we would like to make best use of the reconstructed data
since this may be expensive or impractical to generate arbitrary amounts of.
It is, conversely, likely that generating events from the denominators is comparatively cheap and quick.
We therefore would like to simply concatenate our \texttt{Reco} points, and find the correct
proportions of each \texttt{Gen} distribution to compare it to in order to evaluate the efficiency:

\begin{equation}
    \epsilon(x) = \frac{\rm{Reco^{\alpha}(x) + Reco^{\beta}(x)}}{k_{\alpha}\rm{Gen^{\alpha}(x)} + k_{\beta}\rm{Gen^{\beta}(x)}}
    \label{combined}
\end{equation}

The rest of this note outlines a method for evaluating this efficiency given two reconstructed and generator-level distributions.

\section{Formalism}
Consider a PDF $f(x)$.
Consider the case where the sampling from this PDF is imperfect: i.e. instead of our sample looking like e.g. $f(x)$, it will instead
look like Eq.~\ref{imperfect sampling}:
where $\epsilon(x)$ is some ``efficiency'' function that modifies our underlying distribution.
\begin{equation}
    F(x) = \epsilon(x)f(x)
    \label{imperfect sampling}
\end{equation}

In this paper we wish to extract $\epsilon(x)$; in the case of Eq.~\ref{imperfect sampling}, we can extract $\epsilon(x)$ with our knowledge of $f(x)$ by measuring $F(x)$.

Consider however the case where we have two distributions:
\begin{equation}
    \begin{split}
        F(x) = \epsilon(x)f(x)\\
        G(x) = \epsilon(x)g(x)
    \end{split}
    \label{two_efficiencies}
\end{equation}
These distributions are not true PDFs, since the efficiency function may not be normalised.
Consider now the case where we have different, unknown numbers of samples from each of these distributions.
We reconstruct:
\begin{equation}
    H^{reco}(x) = \alpha\epsilon(x)f(x) + \beta\epsilon(x)g(x)
    \label{combined_pdf}
\end{equation}
where $\alpha$ and $\beta$ are factors that tell us the relative proportion of each distribution in $H(x)$.
These factors may not be known a priori.

It is possible, however, to extract the efficiency: we can write
\begin{equation}
    \begin{aligned}
        C_\alpha & = \alpha\int\epsilon dx                                \\
                 & = \alpha\int\epsilon\frac{f(x)}{f(x)}dx                \\
                 & = \int\left[\alpha\epsilon f(x)\right]\frac{1}{f(x)}dx \\
                 & \approx \sum_{f}\frac{1}{f(x)}
    \end{aligned}
    \label{correction_factor}
\end{equation}
Where the Riemman sum is evaluated over the reconstructed f-type events.
Similarly, we can write down the factor $C_\beta = \sum_g\frac{1}{g(x)}$.

We can then use these sums to evaluate the efficiency:
\begin{equation}
    \begin{aligned}
        \epsilon(x) & \equiv \frac{\alpha\epsilon(x)f(x) + \beta\epsilon(x)g(x)}{\alpha f(x) + \beta g(x)}                    \\
                    & = \frac{\alpha\epsilon(x)f(x) + \beta\epsilon(x)g(x)}{C_\alpha f(x) + C_\beta g(x)} . \int\epsilon(x)dx
    \end{aligned}
    \label{efficiency_expression}
\end{equation}
where we have found the efficiency function up to a multiplicative constant $\int\epsilon(x)dx$ - changing this constant
does not affect the shape of the efficiency.

\section{Example}
Consider two distributions, shown in Figure.~\ref{f, g dists}.
\begin{figure}[h!]
    \begin{minipage}{0.49\textwidth}
        \includegraphics[width=1.1\textwidth]{code/build/f.png}
        \centering
    \end{minipage}
    \begin{minipage}{0.49\textwidth}
        \includegraphics[width=1.1\textwidth]{code/build/g.png}
        \centering
    \end{minipage}
    \caption{PDFs of our two distributions. Note that they are not normalised in this case.}
    \label{f, g dists}
\end{figure}

We also define an efficiency function that will modify the distribution of points sampled from these distributions, seen in
Figure.~\ref{efficiency_plot}:
\begin{figure}[h!]
    \includegraphics[width=0.8\textwidth]{code/build/efficiency.png}
    \centering
    \caption{Our efficiency function.}
    \label{efficiency_plot}
\end{figure}

Sampling from the distributions $f(x)$ and $g(x)$, subject to the above efficiency, gives the sample seen in
Figure.~\ref{samples}:
\begin{figure}[h!]
    \includegraphics[width=0.8\textwidth]{code/build/recoSamples.png}
    \centering
    \caption{A sample taken from the distributions $f(x)$ and $g(x)$ seen in Figure.~\ref{f, g dists},
        subject to the efficiency function seen in Figure.~\ref{efficiency_plot}.}
    \label{samples}
\end{figure}

Our aim is to recover the efficiency function from the distributions seen in Figure.~\ref{samples}, using our knowledge of the PDFs $f$ and $g$.
We first must find the correction factor according to Eq.~\ref{correction_factor}.
This is done by e.g. taking the sum of the inverse probabilities $\frac{1}{f(x)}$ over the sample taken from the $f$-distribution.
These correction factors are then used to construct a final PDF, which attempts to model the PDF combined (green in Figure.~\ref{samples}) sample before
the efficiency was applied.
This PDF is seen in Figure.~\ref{approx combined pdf}.
\begin{figure}[h!]
    \includegraphics[width=0.8\textwidth]{code/build/approx.png}
    \centering
    \caption{Our estimate at the PDF the combined sample was generated from}
    \label{approx combined pdf}
\end{figure}
We then generate a sample from this PDF, seen in Figure~.\ref{sample combined}.
\begin{figure}[h!]
    \includegraphics[width=0.8\textwidth]{code/build/approxSamples.png}
    \centering
    \caption{A sample taken from the combined PDF seen in Figure.~\ref{approx combined pdf}.
        Also shown is the combined sample that we reconstructed; the ratio of these will be used to find the efficiency.}
    \label{sample combined}
\end{figure}

Finally, the efficiency is recovered by taking the ratio between our MC sample and the sample we generated from our combined PDF.
This is shown in Figure.~\ref{Measured efficiency}.

\begin{figure}[h!]
    \includegraphics[width=0.8\textwidth]{code/build/measured_efficiency.png}
    \centering
    \caption{The efficiency we measure compared to the true efficiency.}
    \label{Measured efficiency}
\end{figure}

there is a bias here but i think it's because I don't account for the different PDF normalisations to begin with TODO

\section{Conclusion}
We can find a combined PDF

\end{document}
