\documentclass{article}

\usepackage{amsmath}

\title{Efficiency from Combined Distributions}

\begin{document}

\maketitle
\begin{abstract}
    Combining two distributions is useful
\end{abstract}

\section{Introduction}
Simulations are widely used in particle physics and science in general to test models and predictions.
One application in particular involves using a Monte-Carlo simulation (hereafter referred to as ``MC'')
to estimate the ``detector efficiency'': the function determining whether a given event will be detected.
This efficiency function may have a complicated form: in the case of a two-body decay (e.g. $D^0 \rightarrow K^+\pi^-$)
the efficiency function will be a constant; for more complicated decays it will have a dependence on some possibly
high-dimensional space. For example, the efficiency function for the $D^0\rightarrow K^+\pi^-\pi^+\pi^-$ decay will
vary over a 5-dimensional space.

A common technique for evaluating the efficiency is to compare MC simulation to
what was generated, as seen in Eq.~\ref{efficiency_definition}
\begin{equation}
    \epsilon(x) = \frac{\rm{Reco}(x)}{\rm{Gen}(x)}
    \label{efficiency_definition}
\end{equation}
Where we find our efficiency $\epsilon(x)$ by comparing our reconstructed (MC) and generated distributions.

In some cases we may have two reconstructed and generated distributions, but expect them to obey the same
efficiency function:
\begin{equation}
    \epsilon(x) = \frac{\rm{Reco^\alpha}(x)}{\rm{Gen^\alpha}(x)} = \frac{\rm{Reco^\beta}(x)}{\rm{Gen^\beta}(x)}
    \label{two distributions}
\end{equation}
where our labels $\alpha$ and $\beta$ may refer to e.g. two CP-conjugate decays.

When evaluating the efficiency using Eq.~\ref{two distributions} we would like to make best use of the reconstructed data
since this may be expensive or impractical to generate arbitrary amounts of.
It is, conversely, likely that generating events from the denominators is comparatively cheap and quick.
We therefore would like to simply concatenate our \texttt{Reco} points, and find the correct
proportions of each \texttt{Gen} distribution to compare it to in order to evaluate the efficiency:

\begin{equation}
    \epsilon(x) = \frac{\rm{Reco^{\alpha}(x) + Reco^{\beta}(x)}}{k_{\alpha}\rm{Gen^{\alpha}(x)} + k_{\beta}\rm{Gen^{\beta}(x)}}
    \label{combined}
\end{equation}

The rest of this note outlines a method for evaluating this efficiency given two reconstructed and generator-level distributions.

\section{Formalism}
Consider a PDF $f(x)$.
Consider the case where the sampling from this PDF is imperfect: i.e. instead of our sample looking like e.g. $f(x)$, it will instead
look like Eq.~\ref{imperfect sampling}:
where $\epsilon(x)$ is some ``efficiency'' function that modifies our underlying distribution.
\begin{equation}
    F(x) = \epsilon(x)f(x)
    \label{imperfect sampling}
\end{equation}

In this paper we wish to extract $\epsilon(x)$; in the case of Eq.~\ref{imperfect sampling}, we can extract $\epsilon(x)$ with our knowledge of $f(x)$ by measuring $F(x)$.

Consider however the case where we have two distributions:
\begin{equation}
    \begin{split}
        F(x) = \epsilon(x)f(x)\\
        G(x) = \epsilon(x)g(x)
    \end{split}
    \label{two_efficiencies}
\end{equation}
These distributions are not true PDFs, since the efficiency function may not be normalised.
Consider now the case where we have different, unknown numbers of samples from each of these distributions.
We reconstruct:
\begin{equation}
    H^{reco}(x) = \alpha\epsilon(x)f(x) + \beta\epsilon(x)g(x)
    \label{combined_pdf}
\end{equation}
where $\alpha$ and $\beta$ are factors that tell us the relative proportion of each distribution in $H(x)$.
These factors may not be known a priori.

It is possible, however, to extract the efficiency: we can write
\begin{equation}
    \begin{aligned}
        C_\alpha & = \alpha\int\epsilon dx                                \\
                 & = \alpha\int\epsilon\frac{f(x)}{f(x)}dx                \\
                 & = \int\left[\alpha\epsilon f(x)\right]\frac{1}{f(x)}dx \\
                 & \approx \sum_{f}\frac{1}{f(x)}
    \end{aligned}
    \label{correction_factor}
\end{equation}
Where the Riemman sum is evaluated over the reconstructed f-type events.
Similarly, we can write down the factor $C_\beta = \sum_g\frac{1}{g(x)}$.

We can then use these sums to evaluate the efficiency:
\begin{equation}
    \begin{aligned}
        \epsilon(x) & \equiv \frac{\alpha\epsilon(x)f(x) + \beta\epsilon(x)g(x)}{\alpha f(x) + \beta g(x)}                    \\
                    & = \frac{\alpha\epsilon(x)f(x) + \beta\epsilon(x)g(x)}{C_\alpha f(x) + C_\beta g(x)} . \int\epsilon(x)dx
    \end{aligned}
    \label{efficiency_expression}
\end{equation}
where we have found the efficiency function up to a multiplicative constant $\int\epsilon(x)dx$ - changing this constant
does not affect the shape of the efficiency.

\end{document}

\section{Example}

\section{Conclusion}